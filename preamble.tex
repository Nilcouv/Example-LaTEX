%%%%%%%%%%%%%%%%%% FRENCH NOTATION %%%%%%%%%%%%%%%%%%
%\usepackage[french]{babel} % /!\ BUG L'INSERTION DE VARIABLE POUR LE POSITIONNEMENT DES NOEUDS!assure la bonne césure des mots français et utilise les mots français pour les dates, etc.
%\DecimalMathComma % veille à ce que LaTeX ne place pas automatiquement d'espace après la virgule d'un nombre décimal
\frenchspacing % supprime les espaces additionnels suivant les points finaux
% Ouvrir le fichier frenchb.dtx dans LaTeX et produire le document frenchb.dvi pour plus d'information.

%%%%%%%%%%%%%%%%%% FONTS %%%%%%%%%%%%%%%%%%
%\usepackage{fontspec} 
% permet de définir le type de police de caractère utilisé SEULEMENT EN XeLaTeX et en LuaLaTeX!
%
\usepackage[utf8]{inputenc} 
% encodage dans le fichier source en UTF8 pour admettre les caractères accentués
% En enregistrant le fichier avec extension .tex, il faut choisir l'encodage UTF8.
% De même, en ouvrant ce fichier ("Ouvrir..."), il faut désigner l'encodage UTF8.
\usepackage[T1]{fontenc} % encodage des caractères en 8 caractères binaires dans le fichier TeX produit (puis dans le PostScript ou le PDF)
%\usepackage{lmodern} % cette police de caractères Latin Modern est la plus souvent utilisée dans les encodages en 8 caractères binaires
% La police Latin Modern complète la police originelle Computer Modern encore largement utilisée dans TeX mais qui ne dispose pas des caractères accentués.
%
\usepackage{tgadventor}
% paquet de police de caractère TGadventor
%
\usepackage{sansmath}
% Copie-colle la police de ssf dans la police par défaut
%
%\usepackage[sfdefault]{noto} 
% cette police de caractères est plus lisible (sans serif) que la police Latin Modern, particulièrement pour les titres ou les exercices destinés aux enfants.
%
%\usepackage{kpfonts} 
% It includes a complete set of features including mathematics as well as non-math characters and looks a little more interesting without being distracting.
%
%\usepackage[urw-garamond]{mathdesign} 
% à tester: police Garamond Expert with Math Design
%
%\usepackage{garamondx} % à tester: police Garamond Expert with Math Design

%%%%%%%%%%%%%%%%%% MATH. NOTATION %%%%%%%%%%%%%%%%%%
%\usepackage{amsmath} % permet la création d'équations avec \begin{equation} \end{equation}
\usepackage{amssymb} % pour les ensembles de nombres (\mathbb{R}) et des symboles du type: \rightsquigarrow et \bigstar
\usepackage{amsthm} % for the "proof" environment
%\usepackage{MnSymbol} % pour l'affichage de la flèche courbe \rcurvearrowdown
\usepackage{numprint} % place des espaces entre groupes de 3 chiffres pour les nombres comportant un grand nombre de chiffres
%\usepackage{graphicx}
% permet l'intégration de graphique grâce à la commande \includegraphics
\usepackage{xcolor}
%\usepackage[usenames, dvipsnames]{xcolor} % A DECLARER AVANT TIKZ!
% permet l'intégration de graphique grâce à la commande \includegraphics
% xcolor permet d'afficher des images en couleurs
% le pilote pdftex permet l'intégration d'images au format JPEG, PNG, PDF. Pour utiliser des images EPS, utiliser le pilote dvips
%
\usepackage{tikz} % permet l'intégration des dessins TikZ (les graphiques Geogebra peuvent être exportés au format TikZ)
\usetikzlibrary{%
    matrix,
    arrows,
    arrows.meta,
    bending,
    calc,
    math,
    shapes,
    backgrounds,
    decorations.markings,
    }

\usepackage{pgfplots} % Permet de tracer de graphiques

\usepackage[pdfauthor={Oscar},%
    pdftitle={},%
    bookmarks,colorlinks]{hyperref}

\usepackage{enumitem} % permet de réduire les espaces dans les environnements itemize avec l'option [nosep]
\usepackage{textcomp} % symbole € via commande \texteuro
\usepackage{comment} % pour (dé)commenter plusieurs lignes à l'aide de \includecomment ou \excludecomment
% Il faut modifier ThisComment pour éviter d'interpréter les caractères UTF8 dans les commentaires
% Voir: http://tex.stackexchange.com/questions/159820/comment-sty-and-utf8-encoding
\renewcommand\ThisComment[1]{%
  \immediate\write\CommentStream{\unexpanded{#1}}%
}
\usepackage{cancel} % permet, dans un environnement mathématique, de tracer une barre diagonale sur une expression
%\usepackage{icomma}    %   gestion des espaces après la virgule
