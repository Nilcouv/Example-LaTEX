%%%%%%%%%%%%%%%%%%%%%%%% LEXIQUE %%%%%%%%%%%%%%%%%%%%%%%%
% \pic[fill=green, text=green, draw=black, opacity=.7] at (5,0) {squaroid={width=1, height=1, units=cm, scale=1}};
%%%%%%%%%%%%%%%%%%%%%%%% calibration cross %%%%%%%%%%%%%%%%%%%%%%%%
\tikzset{pics/calcross/.style={code={
    \tikzset{calcross/.cd,#1}
    \def\pv##1{\pgfkeysvalueof{/tikz/calcross/##1}}
    %draw the main coordinate system axes
    \draw[thick,->] (0,0) -- (1,0) node[anchor=north east]{$x$};
    \draw[thick,->] (0,0) -- (0,1) node[anchor=north west]{$y$};
    }},
}

%%%%%%%%%%%%%%%%%%%%%%%% polygone à n côté v.210323 %%%%%%%%%%%%%%%%%%%%%%%%
\tikzset{%
    pics/nfig/.style={%
        code={%
            \tikzset{nfig/.cd,#1}%
            \def\pv##1{\pgfkeysvalueof{/tikz/nfig/##1}}
            \pgfmathtruncatemacro{\NbStep}{(360-\pv{first})/\pv{step}}
            \draw[thick, pic actions, draw, fill=\pv{fill}, rotate=\pv{rotate} ] 
                (\pv{first}:\pv{radius})   % starting point
                \foreach \x [evaluate=\x as \Angle using int(\pv{first}+\pv{step}*\x)] in {0,1,...,\NbStep}{%
                    -- (\Angle:\pv{radius})}                       % loop drawing all the point
                -- cycle                    % close the polygon
            ;
        }
    },
  nfig/.cd,
  radius/.initial=3cm,
  first/.initial=45,
  step/.initial=90,
  fill/.initial=none,
  rotate/.initial=0,
}
%%%%%%%%%%%%%%%%%%%%%%%% SQUAROID %%%%%%%%%%%%%%%%%%%%%%%%
\tikzset{pics/squaroid/.style={%
    code={%
        \tikzset{squaroid/.cd,#1}%
        \def\pv##1{%
            \pgfkeysvalueof{/tikz/squaroid/##1}%
            }
    \draw [solid, pic actions, fill=\pv{fill}, rotate=\pv{rotate}]
    % point o - bas gauche
    (0,0) coordinate (o)
    % point a - bas droite
    (\pv{scale}*\pv{width},0) coordinate (a)
    % point b - haut droite
    (\pv{scale}*\pv{width}+\pv{scale}*\pv{offset}-\pv{scale}*\pv{shrink},\pv{scale}*\pv{height}) coordinate (b)
    % point c - haut gauche
    (0+\pv{scale}*\pv{offset}+\pv{scale}*\pv{shrink},\pv{scale}*\pv{height}) coordinate (c) 
    % connexion entre les points
    (o) 
    -- (a)
    -- (b)
    -- (c)
    -- (o)
    ;
    }},
    squaroid/.cd,
    width/.initial=3,
    height/.initial=2,
    offset/.initial=0,
    shrink/.initial=0,
    scale/.initial=1,
    inner sep/.initial=0pt,
    fill/.initial=none,
    rotate/.initial=0,
}

%%%%%%%%%%%%%%%%%%%%%%%% REGULAR SQUAROID %%%%%%%%%%%%%%%%%%%%%%%%
\tikzset{pics/squaroid regular/.style={code={%
    \tikzset{squaroid regular/.cd,#1}%
    \def\pv##1{\pgfkeysvalueof{/tikz/squaroid regular/##1}}
    \draw[thick, fill=\pv{fill}, pic actions] 
    (45:\pv{radius}) 
    \foreach \x in {135,225,...,359}
        {
            -- (\x:\pv{radius})}%
            -- cycle 
            (90:\pv{radius})%
            ;
        }
    (0,\pv{radius}) coordinate node[above, scale=2.5] {\pv{lab}}
    },
  squaroid_regular/.cd,
  radius/.initial=3cm,
  fill/.initial=none,
}

%%%%%%%%%%%%%%%%%%%%%%%% TRIANGLOID %%%%%%%%%%%%%%%%%%%%%%%%
\tikzset{pics/triangloid/.style={code={%
    \tikzset{triangloid/.cd,#1}%
    \def\pv##1{\pgfkeysvalueof{/tikz/triangloid/##1}}
    \draw [solid, pic actions, fill=\pv{fill}, rotate=\pv{rotate}]
    %%  point o - bas gauche
    (0,0) coordinate (o)
    %% point a - bas droite
    (\pv{scale}*\pv{width},0) coordinate (a)
    %% point b - sommet
    ({\pv{scale}*\pv{offset})},\pv{scale}*\pv{height}) coordinate (b)
    %% jonction entre les points
     (o) 
     -- (a)
     -- (b)
     -- (o)
     ;
  }},
  triangloid/.cd,
  width/.initial=5,
  height/.initial=2,
  offset/.initial=0,
  inner sep/.initial=0pt,
  fill/.initial=none,
  rotate/.initial=0,
  scale/.initial=1,
}

%%%%%%%%%%%%%%%%%%%%%%%% regular TRIANGLOID %%%%%%%%%%%%%%%%%%%%%%%%
\tikzset{pics/triangloid regular/.style={code={%
    \tikzset{triangloid regular/.cd,#1}%
    \def\pv##1{\pgfkeysvalueof{/tikz/triangloid regular/##1}}%
    \draw[thick, pic actions] (-30:\pv{radius}) \foreach \x in {90,210}%
        {
            -- (\x:\pv{radius})}%
            -- cycle
            ;
        }
    },
  triangloid regular/.cd,
  radius/.initial=3cm,
}
%%%%%%%%%%%%%%%%%%%%%%%% TRAPEZOID %%%%%%%%%%%%%%%%%%%%%%%%
\tikzset{pics/trapezoid/.style={code={%
    \tikzset{trapezoid/.cd,#1}%
    \def\pv##1{\pgfkeysvalueof{/tikz/trapezoid/##1}}
    \draw [solid, pic actions]
    %%  point o - bas gauche
    (0,0) coordinate (o)
    %% point a - bas droite
    (\pv{scale}*\pv{width},0) coordinate (a)
    %% point b - haut droite
    ({\pv{scale}*(\pv{widthtop}+\pv{offset})},\pv{scale}*\pv{height}) coordinate (b)
    %% point c - haut gauche
    (\pv{scale}*\pv{offset},\pv{scale}*\pv{height}) coordinate (c)
    %% jonction entre les points
     (o) -- (a) -- (b) -- (c) -- cycle
     ;
  }},
  trapezoid/.cd,
  width/.initial=5,
  widthtop/.initial=3,
  offset/.initial=1,
  height/.initial=2,
  scale/.initial=1,
}
%%%%%%%%%%%%%%%%%%%%%%%% PARALLELOGRAM %%%%%%%%%%%%%%%%%%%%%%%%
%%% /!\ USE SQUAROID - Depreciated version /!\
\tikzset{pics/parallelogram/.style={code={%
    \tikzset{parallelogram/.cd,#1}%
    \def\pv##1{\pgfkeysvalueof{/tikz/parallelogram/##1}}
    \draw [solid, pic actions, fill=\pv{fill}]
    %%  point o - bas gauche
    (0,0) coordinate (o)
    node[label={[below, inner sep=\pv{inner sep}, shift={(0pt,-8pt)}]\pv{labo}}] {}
    %% point a - bas droite
    (\pv{scale}*\pv{width},0) coordinate (a)
    node[label={[below, inner sep=\pv{inner sep}, shift={(0pt,-8pt)}]\pv{laba}}] {}
    %% point b - haut droite
    ({\pv{scale}*(\pv{width}+\pv{offset})},\pv{scale}*\pv{height}) coordinate (b)
    node[label={[above, inner sep=\pv{inner sep}, shift={(0pt,0pt)}]\pv{labb}}] {}
    %% point c - haut gauche
    (\pv{scale}*\pv{offset},\pv{scale}*\pv{height}) coordinate (c)
    node[label={[above, inner sep=\pv{inner sep}, shift={(0pt,0pt)}]\pv{labc}}] {}
    %% jonction entre les points
     (o) 
     -- (a)
     node[midway, label={[below, inner sep=\pv{inner sep}, shift={(0pt,-8pt)}]\pv{labA}}] {}
     -- (b) 
     node[midway, label={[right, inner sep=\pv{inner sep}, shift={(6pt,0pt)}]\pv{labB}}] {}
     -- (c) 
     -- (o)
     ;
  }},
  parallelogram/.cd,
  width/.initial=5,
  offset/.initial=1,
  height/.initial=2,
  labA/.initial=,
  labB/.initial=,
  labo/.initial=,
  laba/.initial=,
  labb/.initial=,
  labc/.initial=,
  inner sep/.initial=0pt,
  scale/.initial=1,
  fill/.initial=none,
}

%%%%%%%%%%%%%%%%%%%% REGULAR PENTAGOID %%%%%%%%%%%%%%%%%%%%%
\tikzset{pics/pentagoid regular/.style={code={%
    \tikzset{pentagoid regular/.cd,#1}%
    \def\pv##1{\pgfkeysvalueof{/tikz/pentagoid regular/##1}}
    \draw[thick, fill=\pv{fill}] (18:\pv{radius}) 
    \foreach \x in {90,162,...,359}
        {-- (\x:\pv{radius})}%
        -- cycle
        ;
        }
    ;
    },
  pentagoid regular/.cd,
  radius/.initial=3cm,
  fill/.initial=none,
}
%%%%%%%%%%%%%%%%%%%% REGULAR HEXAGOID %%%%%%%%%%%%%%%%%%%%%
\tikzset{pics/hexagoid regular/.style={code={%
    \tikzset{hexagoid regular/.cd,#1}%
    \def\pv##1{\pgfkeysvalueof{/tikz/hexagoid regular/##1}}
    \draw[thick, fill=\pv{fill}] (0:\pv{radius}) 
    \foreach \x in {60,120,...,359}
        {-- (\x:\pv{radius})}%
        -- cycle
        ;
        }
    ;
    },
  hexagoid regular/.cd,
  radius/.initial=3cm,
  fill/.initial=none,
}
%%%%%%%%%%%%%%%%%%%%%%%% CROSS %%%%%%%%%%%%%%%%%%%%%%%%
\tikzset{pics/croix/.style={code={%
    \tikzset{croix/.cd,#1}%
    \def\pv##1{\pgfkeysvalueof{/tikz/croix/##1}}
    \draw [solid, pic actions, scale=\pv{scale}, rotate=\pv{rotate}]
    %% point a - base gauche
    (\pv{length},0) coordinate (a)
    %% point b - base droit
    ({\pv{width}-\pv{length}},0) coordinate (b)
    %% point c - centre inférieur droit
    ({(\pv{width}-\pv{length})},\pv{length}) coordinate (c)
    %% point d - branche droite inférieur
    (\pv{width},\pv{length}) coordinate (d)
    %% point e - branche droite supérieur
    (\pv{width},{\pv{height}-\pv{length}}) coordinate (e)
    %% point f - centre supérieur droit
    ({\pv{width}-\pv{length}},{\pv{height}-\pv{length}}) coordinate (f)
    %% point h - sommet droit
    ({\pv{width}-\pv{length}},\pv{height}) coordinate (g)
    %% point i - sommet gauche
    (\pv{length},\pv{height}) coordinate (h)
    %% point j - centre supérieur gauche
    (\pv{length},{\pv{height}-\pv{length}}) coordinate (i)
    %% point k - branche gauche supérieur
    (0,{\pv{height}-\pv{length}}) coordinate (j)
    %% point l - branche gauche inférieur
    (0,\pv{length}) coordinate (k)
    %% point m - centre inférieur gauche
    (\pv{length},\pv{length}) coordinate (l)
    %% jonction entre les points
     (a) -- (b) -- (c) -- (d) -- (e) -- (f) -- (g) -- (h) -- (i) -- (j) -- (k) -- (l) -- cycle
     ;
  }},
  croix/.cd,
  width/.initial=5,
  height/.initial=5,
  length/.initial=2,
  scale/.initial=1,
  rotate/.initial=0,
}

%%%%%%%%%%%%%%%%%%%%%%%% LOSANGE v.20210325 %%%%%%%%%%%%%%%%%%%%%%%%
\tikzset{pics/losange/.style={code={%
    \tikzset{losange/.cd,#1}%
    \def\pv##1{\pgfkeysvalueof{/tikz/losange/##1}}
    \draw [solid, pic actions, rotate=\pv{rotate}, scale=\pv{scale}]
        %%  point a - bas
        (0,0) coordinate (a)
        %% point b - milieu droit
        (\pv{width}/2,{(\pv{height}/2)+\pv{yoffset}}) coordinate (b)
        %% point c - sommet
        (0,\pv{height}) coordinate (c)
        %% point c - haut gauche
        (-\pv{width}/2,{(\pv{height}/2)+\pv{yoffset}}) coordinate (d)
        %% jonction entre les points
        (a) -- (b) -- (c) -- (d) -- cycle
    ;
  }},
  losange/.cd,
  width/.initial=3,
  height/.initial=5,
  yoffset/.initial=0,
  scale/.initial=1,
  rotate/.initial=0,
}